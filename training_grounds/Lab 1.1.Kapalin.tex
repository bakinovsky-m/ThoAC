\documentclass{beamer}

% for themes, etc.
\mode<presentation>
{
%\usetheme{boxes}
\usetheme{Boadilla}
\usecolortheme{seahorse}
%\usetheme{CambridgeUS}
\beamertemplatenavigationsymbolsempty
}

\usepackage{amsmath, amsfonts , amssymb, amsthm}
% \usepackage[cp1251]{inputenc}
% \usepackage{mathtext}
% \usepackage[russian]{babel}
\usepackage[T1,T2A]{fontenc}
\usepackage[utf8]{inputenc}
\usepackage[russian]{babel}

\usepackage{graphicx}
\ifpdf\usepackage{epstopdf}\fi
\usepackage{epsfig}

\theoremstyle{plain}
\newtheorem{theor}{\bf Теорема \thetheor}
\newtheorem{statement}{\bf Утверждение \thestatement}
%%%%%%%%%%%%%%%%%%%%%%%%%%%%%%%%%%%%%%%%%%%%%%%%%%%%%%%%%%%%%%%%%%%%%%%%%%
%%%%%%%%%%%%%%%%%%%%%%%%%%%%%%%%%%%%%%%%%%%%%%%%%%%%%%%%%%%%%%%%%%%%%%%%%%
\theoremstyle{definition}
\newtheorem{defi}{\bf Определение \thedefi}
\newtheorem{task}{\bf Задача  \thetask}
\theoremstyle{remark}
\newtheorem{remark}{\bf Замечание \theremark}
\newtheorem{examp}{\bf Пример}
\theoremstyle{plain}
%%%%%%%%%%%%%%%%%%%%%%%%%%%%%%%%%%%%%%%%%%%%%%%%%%%%%%%%%%%%%%%%%%%%%%%%%%
\newcommand{\eref}[1]{(\ref{#1})}
\newcommand{\const}{\mathop{\mathrm{const}}\nolimits}
\newcommand{\col}{\mathop{\mathrm{col}}\nolimits}
\newcommand{\diag}{\mathop{\mathrm{diag}}\nolimits}
\newcommand{\rank}{\mathop{\mathrm{rank}}\nolimits}
\newcommand{\spec}{\mathop{\mathrm{spec}}\nolimits}
\newcommand{\rang}{\mathop{\mathrm{rang}}\nolimits}
\newcommand{\RE}{\mathop{\mathrm{Re}}\nolimits}
\newcommand{\IM}{\mathop{\mathrm{Im}}\nolimits}
\newcommand{\cnt}[1]{\overline{#1}}
\newcommand{\BF}[1]{\mathbb{#1}}
\newcommand{\IT}[1]{\mathcal{#1}}


% these will be used later in the title page
\title[Основы теории автоматического управления]{Основы теории автоматического управления}
\author{Капалин И.В.}
\institute[\text{МИСиС ИК}]{НИТУ МИСиС\\ Кафедра инженерной кибернетики}


\begin{document}

\section{Main}

\begin{frame}

\begin{center}
{\LARGE
Лабораторная работа №1\\
Стабилизация линейных объектов. Элементы моделирования динамических объектов. }
\end{center}

\end{frame}


\begin{frame}{Задание №1}


1) Стабилизировать методом полиномиальной стабилизации линейный объект с ПФ $W(s)$, т.е. построить регулятор $R(s)$, причем полином замкнутой системы равен $\gamma(s) = (s+1)(s+2)(s+3)$. Регулятор входит в прямую цепь в замкнутом контуре перед $W(s)$.\\
2) В системе Matlab продемонстрировать разницу в блоке Scope (в одной системе координат) в выходах для стабилизированной системы и исходной при соответствующих входах $u(t) = \chi(t)$ и $u(t) = \sin(t)$. Модель составить аналогично рассмотренной в аудитории.

ПФ выбирается в соответствии с номером в списке группы $N$:
$$
W(s) = \frac{(N\mod2)s+(N+7)\mod5+1}{(s+ N\mod5 + 1 )(s- N\mod9)},
$$
где $A \mod M$ означает оператор взятия остатка от деления числа $A$ на число $M$.
Например, $ 9\mod6 = (3+6)\mod6 = 3\mod 6 = 3$.

\end{frame}

\begin{frame}{Задание №2}

1) Промоделировать выход линейного объекта $W(s)$ при гармоническом входном воздействии $u(t) = A\sin(\omega t)$.\\
2) Сравнить полученный выход с установившимся выходом в соответствии со свойством линейных объектов преобразовывать гармонические сигналы. Модель сделать по аналогии с рассмотренной в классе. ПФ $W(s)$ и параметры $A$ и $\omega$ выбираются в соответствии с номером группы.
$$
A = N\mod7+1/2, \quad \omega = N^2,
$$
$$
W(s)= \frac{0.5N}{s^2+(N\mod 5 +1)s+ N\mod 8 +1}.
$$


\end{frame}


\begin{frame}{Задание №3}

1) Получить эквивалентное описание в ПС объекта заданного в виде ОДУ (см. ниже).\\
2) Сравнить выходы обоих описаний при входных воздействиях $u(t) = \chi(t)$ и $u(t) = \sin(t)$.
Начальные условия для ОДУ везде полагать равными нулю.
$$
y^{(N\ mod\ 3+3)} - (N\mod7)\dot y + (N\mod3+1)y = \dot u + (N\mod5-2)u.
$$

\end{frame}


\begin{frame}{Задание №4: требования и объект управления}

1) Используя метод синтеза регулятора по желаемой передаточной функции, определить желаемую ПФ $W_{\text{ж}}(s)$ и регулятор $R(s)$, обеспечивающий
желаемую ПФ. Для этого использовать требования к качеству замкнутой системе ниже и ПФ объекта $W(s)$.

\vspace{0.5cm}

2) Порядок астатизма по возмущению должен быть не меньше 1, по воздействию --- не меньше 2, характеристический полином $\gamma(s)$ таков, что
переходная характеристика входит в $\Delta$-трубку вокруг своего установившегося значения за время меньшее $0.5$ и не выходит после этого из неё, $\Delta=0.05$.
Объект управления описывается ПФ
$$
W(s)= \frac{s-0.5N}{(s^2+(N\mod 5 +1)s+ N\mod 8 +1)(s+0.5N)},
$$
где $N$ --- номер в списке группы.
\end{frame}


\begin{frame}{Задание №4: верификация замкнутой системы}

3) Продемонстрировать, что полученный регулятор действительно придает замкнутой системе заданные показатели качества. Для этого можно
подать на задающее воздействие и возмущение сигналы $g(t) = 2t+1$ и $f(t) = 5$, соответственно (это подтвердит астатизмы).

\vspace{0.5cm}

4) Для проверки времени регулирования необходимо подать на входы $g(t) = \chi(t)$, $f(t) = 0$ и зрительно убедиться, что график входит в $\Delta$-трубку (её, кстати, можно тоже изобразить).

\end{frame}


\end{document}
