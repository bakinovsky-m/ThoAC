\documentclass[../../TAU.tex]{subfiles}
\begin{document}
\chapter{Устойчивость непрерывных систем управления}

    \defi Система является {\it устойчивой}, если после исчезновения того или иного возмущения ее поведение вернется к заданному режиму. В противном случае система считается нейтральной. либо неустойчивой. \par
    Большинство систем являются нелинейными, а линейными являются лишь их приближения. А. М. Ляпунов в 1892 году показал, что в некоторых случаях по первому приближению можно судить об устойчивости нелинейной системы.

    \theor{\it Теоремы Ляпунова об устойчивости нелинейной системы:}
    \begin{enumerate}
        \item Если все корни характеристического уравнения линеаризованной модели являются левыми, то невозмущенное движение соответствующей нелинейной системы асимптотически устойчиво.
        \item Если среди корней характеристического уравнения линеаризованной модели имеется правый корень, то невозмущенное движение соответствующей нелинейной системы неустойчиво.
        \item Случай, когда среди корней характеристического уравнения линеаризованной модели имеются нейтральные корни (корни на мнимой оси), но нет правых корней, называют критическим. В критическом случае по линеаризованной модели нельзя судить об устойчивости невозмущенного движения нелинейной системы. 

    \end{enumerate}

\section{Устойчивость в линейных системах}

    \defi{}Система 
    \begin{equation}\label{LIN_DS}
        \dot x = Ax + bu, x(0) = x_0
    \end{equation} 
    называется 
    {\it асимптотически устойчивой по Ляпунову}, если 
    $x_\text{о}(t)\rightarrow0$ 
    при 
    $t\rightarrow\infty$ 
    и любом $x_0$.\par
    Как известно, в системах вида
    \eref{LIN_DS}
    решение можно представить в виде суммы свободного и вынужденного движений:
    $$
        x(t) = x_\text{о}(t) + x_\text{ч}(t),
    $$
    где $x_\text{ч}(t)$ --- частное решение неоднородного уравнения при 
    $x_0 = 0$, $x_\text{о}(t)$~--- общее решение уравнения при 
    $u\equiv0$.
    \par
    Общее решение $x_\text{о}(t)$ однородного уравнения описывает свободное движение системы управления (т. е. движение при отсутствии внешних воздействий), определяемое только начальными условиями. Частное решение $x_\text{ч}(t)$ описывает вынужденное движение, определяемое внешними воздействиями.

\subsection{Основное условие устойчивости}

    Характеристическое уравнение системы управления совпадает с характеристическим уравнением дифференциальных уравнений и имеет вид
    $\alpha(\lambda)=a_0\lambda^n+a_1\lambda^{n-1}+...+a_n=0$, 
    где $\alpha(s)$ --- 
    характеристический полином. Если 
    $\lambda_i, (i=1,2,...,q)$ ---
    корни характеристического уравнения кратности
    $k_i, (k_1+k_2+...+k_q=n)$, 
    то общее решение однородного уравнения имеет вид 
    $x_\text{o}(t)=\sum_{i=1}^q{Pi(t) e^{\lambda i t}}$, 
    где 
    $P_i(t)=C_1^{(i)}+C_2^{(i)} t+...+C_{ki}^{(i)} t^{k_i-1}$; 
    где 
    $C_{ki}^{(i)}$ --- постоянные   интегрирования. 
    В частном случае, когда все корни простые, 
    $x_\text{ч}(t)=\sum_{i=1}^nCi e^{\lambda i t}$. 
    По правилу Лопиталя можно показать, что 
    $P_i(t) e^{\lambda i t}\rightarrow 0 $
    при 
    $t\rightarrow\infty$ 
    тогда и только тогда, когда действительная часть корня 
    $\lambda_i$
    отрицательна: 
    $Re\ {i}<0$. 

    \defi{\it Основное условие устойчивости.} Для того чтобы система управления была устойчива, необходимо и достаточно, чтобы все корни ее характеристического уравнения имели отрицательную вещественную часть (т.е. лежали в левой полуплоскости). 

\subsection{Необходимое условие устойчивости}

    \theor{\it Необходимое условие устойчивости} Для того чтобы система была устойчива, необходимо, чтобы все коэффициенты ее характеристического уравнения были строго одного знака: 
    $$
        a_0>0,\ a_1>0, ...\ ,\ a_n>0\ \text{или}\ a_0<0,\ a_1<0, ...\ ,\ a_n<0\
    $$
    Если одно из этих условий не выполняется,то система является неустойчивой.

    \begin{proof}
        Представим характеристический полином в виде разложения
        $\alpha(\lambda)=a_0(\lambda-\lambda_1) (\lambda-\lambda_2) ... (\lambda-\lambda_n)$.
        Действительному отрицательному корню
        $\lambda_k = -\alpha_k, (\alpha_k>0)$
        в разложении соответствует множитель
        $\lambda - \lambda_k=\lambda+\alpha_k$.
        Паре комплексно-сопряженных корней с отрицательной вещественной частью 
        $\lambda_l=-\alpha_l+j \beta_l$ 
        и 
        $\lambda_{l+1}=-\alpha_l-j \beta_l$, 
        $(\alpha_l,\ \beta_l > 0)$ 
        соответствует множитель 
        $(\lambda-\lambda_l) (\lambda-\lambda_{l+1}) = (\lambda+\alpha_1 - j \beta_l) (\lambda+\alpha_1 + j \beta_l) = (\lambda+\alpha_l)^2+\beta_l^2$.
        Следовательно если все корни характеристического уравнения имеют отрицательные вещественные части, то характеристический полином может быть представлен как произведение полиномов первой и второй степени с положительными коэффициентами, и соответственно все его коэффициенты при 
        $a_0>0$ будут положительными и при 
        $a_0<0$ --- отрицательными.
    \end{proof}

\section{Алгебраические критерии устойчивости}

    \defi{\it Алгебраическими критериями} устойчивости называются такие  условия, составленные из коэффициентов характеристического уравнения, при выполнении которых система устойчива, а при невыполнении --- неустойчива. При проведении исследования устойчивости следует первоначально проверить выполнение необходимого условия устойчивости.

\subsection{Критерий Гурвица}

    Определим матрицу Гурвица
    $$
        H =\begin{pmatrix}
            a_{n-1} & a_{n-3} & 0 & \ldots & 0 \\
            1 & a_{n-2} & a_{n-4} & \ldots & 0 \\
            \vdots & \vdots & \ddots && \vdots \\
            0 & \ldots & a_{3} & a_1 & 0 \\
            0 & \ldots & a_4 & a_2 & a_0
        \end{pmatrix}\in\BF{R}^{n\times n},
    $$
    \theor[Гурвица] Пусть 
    $a_i>0,\; i=\cnt{0,n-1}$.
    Полином $\alpha(\lambda)$ устойчив тогда и только тогда, когда все главные миноры $\Delta_i$ матрицы Гурвица положительны. \\
    Из коэффициентов характеристического полинома 
    $\alpha(\lambda)=a_0(\lambda-\lambda_1) (\lambda-\lambda_2) ... (\lambda-\lambda_n)$
    составляется определитель n-го порядка:
    $$
        \Delta_i =
        \begin{vmatrix}
            a_{n-1}& \ldots& *\\
            *& \ddots & *\\
            *& \ldots & a_{n-i}
        \end{vmatrix}
    $$

    \examp Исследовать на устойчивость полином\\
    $\alpha(\lambda) = \lambda^4+\lambda^3+5\lambda^2+10\lambda+1$ 
    с помощью критерия Гурвица.

    {\it Решение:}\par
    Необходимое условие устойчивости выполнено, составим матрицу Гурвица.
    $$
        H = 
        \begin{pmatrix}
            10 & 1 & 0 & 0\\
            1 & 5 & 1 & 0 \\
            0 & 10 & 1 & 0\\
            0 & 1 & 5 & 1 \\
        \end{pmatrix}
    $$

    Найдем определители Гурвица: 
    $$
    \Delta_1=10,\ \Delta_2 = 
    \begin{vmatrix}
        10 & 1\\
        1  & 5
    \end{vmatrix}
    =49,\ \Delta_3=
    \begin{vmatrix}
        10 & 1 & 0\\
        1  & 5 & 1\\
        0 & 10 & 1\\
    \end{vmatrix}
    =-51<0.
    $$
    Система неустойчива.

    \examp Исследовать на устойчивость полином\\
    $\alpha(\lambda) = \lambda^4+5\lambda^3+7\lambda^2+11\lambda+8$ 
    с помощью критерия Гурвица.

    {\it Решение:}\par
    Необходимое условие устойчивости выполнено, составим матрицу Гурвица.
    $$
        H = 
        \begin{pmatrix}
            11 & 5 & 0 & 0\\
            8 & 7 & 1 & 0 \\
            0 & 11 & 5 & 0\\
            0 & 8 & 7 & 1 \\
        \end{pmatrix}
    $$

    Найдем определители Гурвица: 
    $$
    \Delta_1=11,\ \Delta_2 = 
    \begin{vmatrix}
        11 & 5\\
        8  & 7
    \end{vmatrix}
    =77 - 40=37,\ \Delta_3=
    \begin{vmatrix}
        11 & 5 & 0\\
        8  & 7 & 1\\
        0 & 11 & 5\\
    \end{vmatrix}
    =64,\ \Delta_4=1 \cdot \Delta_3=64.
    $$
    Так как все определители Гурвица больше нуля, то полином устойчив. 

\subsection{Критерий Рауса}

    \theor[Рауса] Для того чтобы полином 
    $\alpha(\lambda)$ был устойчив, необходимо и достаточно, чтобы все элементы первого столбца таблицы Рауса при $a_0 > 0$ были положительны: 
    $C_{k1}>0,\ k=1, 2,..., n+1$.


    \begin{table}[h]
        \caption{Таблица Рауса ($r_k = c_{k-2,1}/c_{k-1,1}$)}
        {\small
        \begin{tabular}{|c|c|c|c|c|}
            \hline
              % after \\: \hline or \cline{col1-col2} \cline{col3-col4} ...
                № & 1 & 2 & 3 & \ldots \\
            \hline
                1 & 
                $c_{1,1}=a_n=1$ & 
                $c_{1,2}=a_{n-2}$ & 
                $c_{1,3} = a_{n-4}$ & 
                \ldots \\
            \hline
                2 &
                $c_{2,1} = a_{n-1}$ &
                $c_{2,2} = a_{n-3}$ & 
                $c_{2,3} = a_{n-5}$ & 
                \ldots \\
            \hline
                3 & 
                $c_{3,1} = c_{1,2} - r_3c_{2,2}$ & 
                $c_{3,2} = c_{1,3} - r_{3}c_{2,3}$ & 
                $c_{3,3} = c_{1,4}-r_{3}c_{2,4}$ & 
                \ldots \\
            \hline
                4 & 
                $c_{4,1} = c_{2,2} - r_4c_{3,2}$ & 
                $c_{4,2} = c_{2,3}-r_4c_{3,3}$ & 
                $c_{4,3} = c_{2,4} - r_4c_{3,4}$ & 
                \ldots \\
            \hline
                \ldots & \ldots & \ldots & \ldots & \ldots \\
            \hline
                $n+1$ & 
                $c_{n+1,1} = c_{n-1,2} - r_{n+1}c_{n,2}$ & 
                & & \\
            \hline
        \end{tabular}
        }
    \end{table}
    \FloatBarrier

    \remark{О таблице Рауса}
    \begin{enumerate}
        \item В таблице Рауса ровно $n+1$ строка и 
        $\lceil\frac{n+1}{2}\rceil$ 
        столбец;
        \item Строки вычитаются целиком одна из другой;
        \item в позициях, где не хватает элемента  (или не был записан результат) пишется нуль.
    \end{enumerate}

    \examp Исследовать на устойчивость полином\\
    $\alpha(\lambda) = \lambda^5+2\lambda^4+8\lambda^3+4\lambda^2+5\lambda+6$ 
    с помощью критерия Рауса.

    {\it Решение:}\par
    Необходимое условие устойчивости выполнено, воспользуемся критерием Рауса. Вычислим элементы таблицы Рауса.
    \begin{align*}
        c_{1,1}&=a_5=6,\ c_{1,2}=a_3=4,\ c_{1,3}=a_1=2, \\
        c_{2,1}&=a_4=5,\ c_{2,2}=a_2=8,\ c_{2,3}=a_0=1, \\
        c_{3,1}&=c_{1,2}-\frac{c_{1,1}}{c_{2,1}}=4-\frac{6}{5}\cdot 8 = -\frac{28}{5}...
    \end{align*}
    Так как элемент $c_{3,1}$ отрицателен, полином не устойчив 


    \examp Исследовать на устойчивость полином\\
    $\alpha(\lambda) = \lambda^4+4\lambda^3+5\lambda^2+7\lambda+3$ 
    с помощью критерия Рауса.

    {\it Решение:}\par
    Необходимое условие устойчивости выполнено, воспользуемся критерием Рауса. Составим таблицу Рауса:
    \begin{center}
        \begin{tabular}[h]{|c|c|c|}
            \hline
            3 & 5 & 1 \\
            \hline
            7 & 4 & 0 \\
            \hline
            $\frac{23}{7}$ & 1 & 0 \\
            \hline
            $\frac{43}{23}$ & 1 & 0 \\
            \hline
            1 & 0 & 0 \\
            \hline
        \end{tabular}
    \end{center}
    Все элементы первого столбца положительны, следовательно, система устойчива.

\section{Частотные критерии устойчивости}

    \defi{\it Частотными критериями устойчивости} называются условия устойчивости, основанные на построении частотных характеристик и так называемой кривой Михайлова.
    \defi{\it Кривая Михайлова} --- годограф характеристического вектора, т.к. кривая, которая описывает характеристический вектор при изменении частоты от $0$ до $\infty$.
    \defi{\it Характеристический вектор} --- выражение 
    $\alpha(j \omega)=a_0(j \omega)^n+a_1(j \omega)^{n-1}+...+a_n$, при $\lambda=j \omega$ в характеристический полином $\alpha(\lambda)$.

\subsection{Принцип аргумента}

    \theor[Принцип аргумента] Пусть полином $\alpha(\lambda)$ имеет $l$ корней в правой полуплоскости и $(n-l)$ --- в  левой (на мнимой оси корней нет). Тогда 
    $\left.\Delta arg \alpha(j\omega)\right|^{\infty}_{\omega=0} = (n-2l)\frac{\pi}{2}$.
    \begin{proof}
        Если разложить полином $\alpha(\lambda)$ на элементарные множители и сделать постановку $\lambda = j \omega$, то получим
        $$
            \alpha(j \omega)=a_0 (j \omega-\lambda_1) (j \omega-\lambda_2) ... (j \omega-\lambda_n),
        $$
        где $\lambda_i\ (i=1,2,...,n)$ ---
        нули полинома $\alpha(\lambda)$.
        Из этого соотношения получим 
        $arg \alpha(\j \omega) = \sum_{i=1}^{n}arg(j \omega-\lambda_i)$ и соответственно
        $$
            \Delta arg\ \alpha(j \omega)=\sum_{i=1}^{n}\Delta \psi_i.
        $$
        Здесь $\Delta \psi_i$ --- приращение аргумента множителя $j \omega-\lambda_i$ при изменении частоты $\omega$ от $0$ до $\infty$.\par
        Найдем $\Delta \psi_i$ отдельно для случаев, когда $\lambda_i$ --- комплексное число.

        \begin{enumerate}
            \renewcommand{\labelenumi}{\asbuk{enumi})}
            \item 
                $\lambda_i = \alpha_i$, $\alpha_i$ ---
                вещественное число. В этом случае
                $$
                    \psi_i(\omega)=arg(j \omega-\alpha_i)=-\arctg{\frac{\omega}{\alpha_i}},
                $$
                \begin{align*}
                \psi_i(0)=-\arctg{0}=0,\ \psi_i(\infty)=
                    \begin{cases}
                        &\frac{\pi}{2}\ \text{при}\ \alpha_i<0,\\
                        -&\frac{\pi}{2}\ \text{при}\ \alpha_i>0,
                    \end{cases}
                \end{align*}
                \begin{align*}
                \Delta\psi_i=\psi_i(\infty)-\psi_i(0)=
                    \begin{cases}
                        &\frac{\pi}{2}\ \text{при}\ \alpha_i<0,\\
                        -&\frac{\pi}{2}\ \text{при}\ \alpha_i>0.
                    \end{cases}
                \end{align*}

                Таким образом, если вещественный нуль является левым ($\alpha_i<0$), приращение 
                $\Delta \psi_i = \frac{\pi}{2}$; 
                если правым ($\alpha_i>0$), приращение $\Delta \psi_i=-\frac{\pi}{2}$
            \item
                $\lambda_i=\alpha_i+ j \beta_i, \alpha_i, \beta_i$ --- вещественные числа. В этом случае существует комплексно-сопряженный нуль 
                $\lambda_{i+1}=\alpha_i+ j \beta_i.$ 
                Приращения множетилей, соответствующих этим нулям, определяются следующим образом:
                $$
                    \psi_i(\omega)=arg(j \omega-\alpha_i-j \beta_i)=\arctg{\frac{\omega-\beta_i}{\alpha_i}},
                $$
                \begin{align*}
                    \psi_i(0)=-\arctg(\frac{\beta_i}{\alpha_i}),\ \psi_i=
                    \begin{cases}
                        &\frac{\pi}{2}\ \text{при}\ \alpha_i<0 ,\\
                        -&\frac{\pi}{2}\ \text{при}\ \alpha_i>0,
                    \end{cases}
                \end{align*}
                \begin{align*}
                    \Delta \psi_i=\psi_i(\infty)-\psi_i(0)=
                    \begin{cases}
                        &\frac{\pi}{2}+\arctg{\frac{\omega-\beta_i}{\alpha_i}}\ \text{при}\ \alpha_i<0 ,\\
                        -&\frac{\pi}{2}+\arctg{\frac{\omega-\beta_i}{\alpha_i}}\ \text{при}\ \alpha_i>0.
                    \end{cases}
                \end{align*}

                Аналогично получаем
                \begin{align*}
                    \Delta \psi_{i+1}=
                    \begin{cases}
                        &\frac{\pi}{2}+\arctg{\frac{\omega-\beta_i}{\alpha_i}}\ \text{при}\ \alpha_i<0 ,\\
                        -&\frac{\pi}{2}+\arctg{\frac{\omega-\beta_i}{\alpha_i}}\ \text{при}\ \alpha_i>0.
                    \end{cases}
                \end{align*}
                Отсюда получаем суммарное приращение 
                $\Delta \psi_i + \Delta \psi_{i+1}=2\cdot(\frac{\pi}{2})$,
                если комплексно-сопряженные нули левые ($\alpha_i<0$), и $\Delta \psi_i + \Delta \psi_{i+1}=-2\cdot(\frac{\pi}{2})$,
                если комплексно-сопряженные нули левые ($\alpha_i>0$).
        \end{enumerate}

        Так как комплексно-сопряженные числа отличаются только мнимой частью, они оба являются левыми или оба являются правыми.
        Поэтому и в случае комплексных нулей <<в среднем>> на каждый левый нуль приходится приращение $\frac{\pi}{2}$, на каждый правый корень --- приращение {$-\frac{\pi}{2}$}.\par
        Таким образом, если полином имеет $l$ правых нулей и $n-l$ левых нулей, то при изменении частоты от $0$ до $\infty$, приращение есть
        $$
            \Delta \alpha(j \omega) = l\cdot(-\frac{\pi}{2})+(n-l)\frac{\pi}{2} = (n-2 l)\frac{\pi}{2},
        $$
        что и требовалось доказать.
    \end{proof}

\subsection{Критерий Михайлова}

    \theor[Критерий Михайлова] 
    Полином $\alpha(\lambda)$ устойчив тогда  и только тогда, когда годограф Михайлова $\alpha(j\omega)$ начинается на положительной вещественной оси и последовательно проходит все $n$ квадрантов в положительном направлении (против часовой стрелки).

    \begin{proof}
        Из принципа аргумента следует, что если все нули характеристического полинома левые, то приращение аргумента характеристического вектора есть $\Delta arg \alpha(j \omega)=\frac{n \pi}{2}$.
    \end{proof}
    %%%%%%%%%%%%%%%%%%%%%%%%
    %%% Вставить годогаф %%%
    %%%%%%%%%%% и %%%%%%%%%%
    %%%%%%%% пример %%%%%%%%
    %%%%%%%%%%%%%%%%%%%%%%%%

\subsection{Критерий Найквиста}

    До этого все критерии формулировались для разомкнутых систем. Следующий критерий формулируется для замкнутой системы с передаточной функцией $W(s)$  в прямой цепи (обратная связь отрицательная).
    \theor[Найквист] Предположим, что разомкнутая система не имеет полюсов на мнимой оси.
    Замкнутая система устойчива тогда и только тогда, когда АФЧХ (годограф Найквиста) разомкнутой системы {\it охватывает} точку $(-1, 0j)$ в {\it положительном направлении} $\frac{l}{2}$ раз, где $l$ --- число правых полюсов ПФ разомкнутой системы.
    \defi{\it Годографом Найквиста} называется кривая, описываемая концом вектора $W(j\omega)$ на комплексной плоскости при изменении $\omega$ от $0$ до $\infty$.

    При наличии мнимых корней у разомкнутой системы формулировка критерия почти не меняется.

    \theor[Найквист]
    Замкнутая система устойчива тогда и только тогда, когда АФЧХ не проходит через точку $(-1,0j)$ и $\left.\Delta Arg(1+W(j\omega))\right|^{\infty}_0 = \frac{\pi}{2}(2l+m)$, где $l$ --- число правых полюсов ПФ разомкнутой системы, $m$ --- число мнимых полюсов ПФ разомкнутой системы.

\section{Условие граничной (маргинальной) устойчивости}

    \defi{Говорят, что линейная система находится на границе устойчивости, если среди корней характеристического уравнения есть чисто мнимые (нейтральные) и нет правых корней.}

\subsection{Граничный коэффициент}

    Пусть дана ПФ разомкнутой системы вида 
    $W(s) =  \frac{k}{s^v}W_0(s), \quad W(0) = 1$. 
    Тогда устойчивость замкнутой системы может зависеть от коэффициента $k$. Такой коэффициент $k=k_\text{г}$, при котором система находится на границе устойчивости, называется {\it граничным коэффициентом}.

    \begin{theor}[Условие граничной устойчивости Найквиста.]
        Замкнутая система находится на границе устойчивости, если АФЧХ разомкнутой системы проходит через точку $(-1,0j)$ и при малой её деформации выполняется  критерий устойчивости Найквиста.
    \end{theor}
    Примером маргинальной замкнутой системы является система с ПФ разомкнутой системы $W(s) = \frac{5}{s^2-1}$. Очевидно, $W(j\omega_0) = -1+0j$, при $\omega_0 = 2$.

\subsection{Устойчивость систем с чистым запаздыванием}

    \defi{\it Чистое запаздывание} --- это транспортное запаздывание, т.е. задержка в реакции выхода ОУ на воздействие (например, душ в ванной, а также системы с ленточными конвейерами). ПФ звена чистого запаздывания имеет вид 
    $W(\lambda) = e^{-\tau \lambda}$. 
    Разомкнутую систему с запаздыванием запишем в виде
    $$
    W_\tau(\lambda) = e^{-\tau \lambda} W(\lambda), \quad W(\lambda) = \frac{B(\lambda)}{A(\lambda)},
    $$
    где $A(\lambda), B(\lambda)$ --- полиномы.
    Для исследования устойчивости такой системы может быть использован критерий Найквиста, формулировка которого практически остается без изменения.\par
    Частотная передаточная функция, амплитудная и фазовая частотные функции разомкнутой системы имеют вид
    \begin{align*}
        W_\tau(j \omega)=W(j \omega)e^{-j \tau \omega},\ & 
        W(j \omega)=R(j \omega)/S(j \omega),\\
        \left|W_\tau(j \omega)\right|=\left|W(j \omega)\right|,\ 
        &\varphi_\tau(\omega)=\varphi(\omega)-\tau \omega,
    \end{align*}
    где 
    $\varphi_\tau(\omega)=arg W_\tau(j \omega),\ \varphi(\omega = arg W(j \omega)$\\
    Легко видеть, что АФЧХ $W_\tau(j \omega)$ отличается от АФЧХ $W(j_\omega)$ только сдвигом по фазе на $-j\omega$.

\section{Понятие грубости (робастности) САУ}

    Любая реальная САУ подвержена различным воздействиям, которые могут опускаться при исследовании из-за их незначительности. К ним относятся старение/разрушение, повреждения, погрешности измерений параметров ОУ и прочее. Это приводит к необходимости требования от САУ грубости (робастности), т.е. чтобы САУ работала не только при конкретно данных значениях параметров ОУ, но и при всех предполагаемых/возможных значениях.

    \defi Пусть дано 
    $\IT{A}\subset\BF{R}^{1\times(n+1)}$ --- 
    множество возможных значений коэффициентов полинома 
    $\alpha(s) = \sum_{i=1}^{n}a_is^i, \; a = (a_n,\; \ldots,\; a_0)$. 
    Полином $\alpha(s)$ называется {\it робастно устойчивым} в $\IT{A}$, если он устойчив для любого $a\in \IT{A}$.

    \begin{theor}[Необходимое условие устойчивости для робастной устойчивости.]
        Все значения коэффициентов полинома $\alpha(\lambda)$ из множества $\IT{A}$ должны быть одного знака (в данном изложении --- больше нуля).
    \end{theor}

\subsection{Критерий Харитонова}

    Пусть множество $\IT{A}$ является параллелепипедом в $\BF{R}^{1\times (n+1)}$:
    $$
        \IT{A}=\left\{a: \underline{ a}_i\leqslant a_i \leqslant \overline a_i, \; i=0,\ldots,n\right\}.
    $$
    Здесь $\underline{a}_i$ и $\overline a_i$ ---
    Максимальное и минимальное значения коэффициента $a_i\ (i=0,\dots,n)$

    \begin{theor}[Критерий Харитонова]
        Полином $\alpha(s)$ робастно устойчив в $\IT{A}$ тогда и только тогда, когда устойчивы все полиномы Харитонова:
        \begin{enumerate}
            \item 
                $k_1(s) = \overline{a}_n s^n + \underline{ a}_{n-1} s^{n-1} + \underline{ a}_{n-2} s^{n-2} + \cdots$\\
            \item 
                $k_2(s) = \overline a_n s^n + \overline a_{n-1} s^{n-1} + \underline{ a}_{n-2} s^{n-2} + \cdots$\\
            \item 
                $k_3(s) = \underline{ a}_n s^n + \overline a_{n-1} s^{n-1} + \overline a_{n-2} s^{n-2} + \cdots$\\
            \item 
                $k_4(s) = \underline{ a}_n s^n + \underline{ a}_{n-1} s^{n-1} + \overline a_{n-2} s^{n-2} + \cdots$
        \end{enumerate}

    \end{theor}

    \begin{proof}
        Так как по определению робастной устойчивости характеристический полином должен быть устойчивым при всех значениях $a\in \IT{A}$, то должны быть устойчивыми и полиномы Харитонова как характеристические полиномы, соответствующие четырем различным значениям $а$ из множества $\IT{A}$.
    \end{proof}

    \examp Исследовать на робастную устойчивость полином $\alpha(s) = s^4 + a_3 s^3 + 2 s^2 + a_1 s + a_0$ в области $1\lse a_3 \lse 4$,\ $ 2 \lse a_1 \lse 3$ $3\lse a_0\lse5$.

    {\it Решение}
    В данном случае 
    $$
        \IT{A} = \left\{a: a_0=1,\ 2\lse a_1 \lse 3,\ a_2 = 2, 1\lse a_3\lse 4,\ a_4=1\right\}
    $$
    \begin{align*}
        \underline{a_0}=\overline{a_0}&=1,\ 
        \underline{a_1}=2,\ \overline{a_1}=3,\
        \underline{a_2}=\overline{a_2}=2,\\
        \underline{a_3}&=1,\ \overline{a_3}=4,\ 
        \underline{a_4}=\overline{a_4}=1
    \end{align*}

    Так как $n=4$, то необходимо и достаточно, чтобы были устойчивы полиномы Харитонова 
    ${K}_1(s)$ и ${K}_2(s)$.

    Таким образом имеем:
    \begin{align*}
        {K}_1(s):\ &\overline{a_4},\ \underline{a_3},\ \underline{a_2},\ \overline{a_1},\ \overline{a_0},\\
        {K}_2(s):\ &\overline{a_4},\ \overline{a_3},\ \underline{a_2},\ \underline{a_1},\ \overline{a_0},\\
        {K}_1(s)= &s^4+s^3+2s^2+3s+1,\\
        {K}_2(s)= &s^4+4s^3+2s^2+2s+1.
    \end{align*}
    $$
        H_1=
        \begin{pmatrix}
            3 & 1 & 0 & 0 \\
            1 & 2 & 1 & 0 \\
            1 & 3 & 1 & 0 \\
            0 & 1 & 2 & 1 \\
        \end{pmatrix}
    $$
    $$
        H_2=
        \begin{pmatrix}
            2 & 4 & 0 & 0 \\
            1 & 2 & 1 & 0 \\
            1 & 3 & 4 & 0 \\
            0 & 1 & 2 & 1 \\
        \end{pmatrix}
    $$

    Необходимое условие устойчивости для обоих полиномов выполняется. По критерию Льенара-Шипара для $n=4$ $\Delta_3>0$.\\
    Проверим для ${K}_1(s)$ и ${K}_2(s)$:
    \begin{align*}
        {K}_1(s)&:\ \Delta_3=3\cdot(-1)-1\cdot(0)=-3,\\
        {K}_2(s)&:\ \Delta_3=2\cdot6-1\cdot6=12-16=-4
    \end{align*}

    Полиномы ${K}_1(s)$ и ${K}_2(s)$ являются неустойчивыми, значит полином $\alpha(s)$ неустойчив

    \examp Исследовать на робастную устойчивость:
    $$
    W=\frac{b_0s+b_1}{s^4+a_1s^3+a_2s^2+a_3s+a4}
    $$
    \begin{align*}
        3\lse\ &a_1\lse4, & 10\lse a_2&\lse15, \\
        0{,}5\lse\ &a_3\lse1, & b_0&=2, \\
        1{,}2\lse\ &a_4\lse1{,}4, & b_1&=-1.
    \end{align*}

    {\it Решение}
    Характеристический полином замкнутой системы имеет вид:
    $$
        \alpha(s)=e_0*s^4+e_1s^3+e_2s^2+e_3s+e_4,
    $$    
    где 
    $e_0=1,\ e_1=a_1,\ e+2 =a_2,\ e_3=a_3+b_0,\ e_4=a_4+b_1$

    Таким образом имеем:
    \begin{align*}
        \underline{e_0}=\overline{e_0}&=1,\ 
        \underline{e_1}=3,\ \overline{e_1}=4,\
        \underline{e_2}=10,\ \overline{e_2}=15,\\
        \underline{e_3}&=2{,}5,\ \overline{e_3}=3,\ 
        \underline{e_4}=0{,}2,\ \overline{e_4}=0{,}4
    \end{align*}

    Так как $n=4$, то необходимо и достаточно, чтобы были устойчивы полиномы  Харитонова 
    ${K}_1(s)$ и ${K}_2(s)$:
    \begin{align*}
        K_1(s)&=s^4+3s^3+10s^2+3s+0{,}4\\
        K_2(s)&=s^4+4s^3+10s^2+2{,}5s+0{,}4
    \end{align*}

    По критерию Льенара-Шипара проверим $\Delta_3$:
    $$
        H_1=
        \begin{pmatrix}
            3 & 3 & 0 & 0 \\
            0{,}4 & 10 & 0 & 0 \\
            0 & 3 & 1 & 0 \\
            0 & 0{,}4 & 10 & 1 \\
        \end{pmatrix}
    $$
    $$
        \Delta_3=3\cdot30-3\cdot0=90>0
    $$
    $$
        H_2=
        \begin{pmatrix}
            2{,}5 & 4 & 0 & 0 \\
            0{,}4 & 10 & 1 & 0 \\
            0 & 2{,}5 & 4 & 0 \\
            0 & 0{,}4 & 10 & 1 \\
        \end{pmatrix}
    $$
    $$
        \Delta_3=2{,}5\cdot40-4\cdot1{,}6=93{,}6>0
    $$

    Так как полиномы Харитонова устойчивы, то замкнутая система робастно устойчива.



\end{document}